\documentclass[12pt]{article}
\usepackage[]{fullpage}
\title{Chapter 16: Notes}
\author{}
\newcommand\T{\rule{0pt}{2.6ex}}       % Top strut
\newcommand\B{\rule[-1.2ex]{0pt}{0pt}} % Bottom strut
\begin{document}
    \maketitle

    \noindent\rule{\textwidth}{0.4pt}
    \section*{Section 16.1}
    \noindent\rule{\textwidth}{0.4pt}
        \subsection*{Basic Properties of $F$-Curves}
            \begin{description}
                \item[Property 1:] The total area under an $F$-curve equals 1.
                \item[Property 2:] An $F$-curve starts at 0 on the horizontal axis and
                extends indefinitely to the right, approaching, but never touching, the
                horizontal axis as it does so.
                \item[Property 3:] An $F$-curve is right-skewed. 
            \end{description}

    \section*{Section 16.2}
    \noindent\rule{\textwidth}{0.4pt}
        \subsection*{Assumptions (Conditions) for One-Way ANOVA}
            \begin{description}
                \item[Simple random samples:] The samples taken from the populations under
                consideration are simple random samples. 
                \item[Independent samples:] The samples taken from the populations under
                consideration are independent of one another.
                \item[Normal populations:] For each population, the variable under
                consideration is normally distributed.
                \item[Equal standard deviations:] The standard deviations of the variable
                under consideration are the same for all the populations.   
            \end{description}
        \subsection*{Mean Squares and $F$-Statistic in One-Way ANOVA}
            \begin{description}
                \item[Treatment mean square, $MSTR$:] The variation among the sample means:
                $MSTR = SSTR/(k-1)$, where $SSTR$ is the treatment sum of squares and $k$ is
                the number of populations under consideration.
                \item[Error mean square, $MSE$:] The variation within the sample: $MSE = SSE/
                (n-k)$, where $SSE$ is the error sum of squares and $n$ is the total number
                of observations.
                \item[$f$-Statistic, $F$:] The ratio of the variation among the sample means
                variation within the samples: $F = MSTR/MSE$.  
            \end{description}

    \section*{Section 16.3}
    \noindent\rule{\textwidth}{0.4pt}
        \subsection*{Distribution of the $F$-Statistic for One-Way ANOVA}
            Suppose that the variable under consideration is normally distributed on each of
            k populations and that the population standard deviations are equal. Then, for
            independent samples from the $k$ populations, the variable
            \begin{center}
                \[
                    F = \frac{MSTR}{MSE}    
                \]
            \end{center}
            has the $F$-distribution with $df = (k-1, n-k)$ if the null hypothesis of equal
            population means is true. Here, $n$ denotes the total number of observations.
        \subsection*{One-Way ANOVA Identity}
            The total sum of squares equals the treatment sum of squares plus the error sum
            of squares: $SST = SSTR + SSE$. $SST$ measures the total variation among all the
            sample data. The total variation among all the sample data can be partitioned
            into two components, one representing variation among the sample means and the
            other representing variation within the samples.
        \subsection*{Sums of Squares in One-Way ANOVA}
            For a one-way ANOVA of $k$ population means, the defining and computing
            formulas for the three sums of squares are as follows.
\begin{table}[h!]
    \centering
    \begin{tabular}{l|l|l}
        Sum of squares&   Defining formula        &   Computing formula \B \\
        \hline
        Total, $SST$  &   $\sum(x_i-\bar{x})^2$   &   $\sum x_i^2 - (\sum x_i)^2/n)$ \T\B \\
        Treatment, $SSTR$ &   $\sum n_j(\bar{x_j}-\bar{x})^2$ &   $\sum(T_j^2/n_j)-(\sum x_i)^2/n$ \T\B \\
        Error, SSE    &   $\sum(n_j-1)s_j^2$  &   $SST-SSTR$ \T \\
    \end{tabular}
\end{table}
            In this table, we used the notation
            \begin{center}
                $n$ = total number of observations,         \newline
                $\bar{x}$ = mean of all $n$ observations;   \newline
            \end{center}
            and, for $j$ = 1, 2,$\dots$,$k$,
            \begin{center}
                \begin{description}
                    \item[$n_j$] = size of sample from Population $j$          
                    \item[$\bar{x}_j$] = mean of sample from Population $j$    
                    \item[$s_j^2$] = variance of sample from Population $j$    
                    \item[$T_j$] = sum of sample data from Population  $j$     
                \end{description}
            \end{center}
            Note that a summation involving a subscript $i$ is over all $n$ observations;
            one involving a subscript $j$ is over the $k$ populations.
        \subsection*{One-Way ANOVA Test}
            \begin{description}
                \item[Purpose] To perform a hypothesis test to compare $k$ population
                means, $\mu_1, \mu_2,\dots,\mu_k$ 
                \item[Assumption]
                \begin{enumerate}
                    \item Simple random samples
                    \item Independent samples
                    \item Normal populations
                    \item Equal population standard deviations
                \end{enumerate} 
                \item[Step 1] The null and alternative hypotheses are, respectively,
                \begin{center}
                    \[
                        H_0: \mu_1 = \mu_2 = \dots = \mu_k
                    \]
                    $H_a$: Not all the means are equal.
                \end{center} 
                \item[Step 2] Decide on the significance level, $\alpha$.
                \item[Step 3] Compute the value of the test statistic  
                \begin{center}
                    \[
                        F = \frac{MSTR}{MSE}    
                    \]
                \end{center}
                and denote that value $F_0$. To do so, construct a one-way ANOVA table:
\begin{table}[h!]
    \centering
    \begin{tabular}{l|l|l|l|l}
        Source&$df$&$SS$&$MS = SS/df$&$F-statistic$ \B \\
        \hline
        Treatment&$k-1$&$SSTR$&$MSTR=\frac{SSTR}{k-1}$&$F=\frac{MSTR}{MSE}$ \T\B \\
        Error&$n-k$&$SSE$&$MSE=\frac{SSE}{n-k}$& \T\B \\
        Total&$n-1$&$SST$&& \T \\
    \end{tabular}
    \end{table}
\end{description}
    
\end{document}



