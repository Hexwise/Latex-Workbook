\documentclass[12pt]{article}
\usepackage[]{fullpage}
\title{Chapter 2: Notes}
\author{}
\begin{document}
    \maketitle

    \noindent\rule{\textwidth}{0.4pt}
    \section*{Section 2.1}
    \noindent\rule{\textwidth}{0.4pt}
        \subsection*{Variable}
            \begin{description}
                \item[Variable:] A characteristic that varies from one person or thing to
                another.
                \item[Qualitative varialbe:] A nonnumerically valued variable.
                \item[Quantitative variable:] A numerically valued variable.
                \item[Discrete variable:] A quantitative variable whose possible values can
                be listed. In particular, a quantitative variable whose possible values can
                be listed. In particular, a quantitative variable with only a finite number
                of possible values is a discrete variable.
                \item[Continuous variable:] A quantitative variable whose possible values
                form some interval of numbers.    
            \end{description}
            A discrete variable usually involves a count of something, whereas a continuous
            variable usually involves a measurement of something.

        \subsection*{Data}
            \begin{description}
                \item[Data:] Values of a variable.
                \item[Qualitative data:] Values of a qualitative variable.
                \item[Quantitative data:] Values of a quantitative variable.
                \item[Discrete data:] Values of a discrete variable.
                \item[Continuous data:] Values of a continuous variable.    
            \end{description}
            Data are classified according to the type of variable from which they were
            obtained.
            
    \section*{Section 2.2}
    \noindent\rule{\textwidth}{0.4pt}
        \subsection*{Frequency Distribution of Qualitative Data}
            A frequency distribution of qualitative date is a listing of the distinct
            values and their frequency. A frequency distribution provides a table of the
            values of the observations and how often they occur.
        \subsection*{Frequency Distribution of Qualitative Data Procedure}
            \begin{description}
                \item[Step 1] List the distinct values of the observations in the data set
                in the first column of a table.
                \item[Step 2] For each observation, place a tally mark in the second column
                of the table in the row of the appropriate distinct values.
                \item[Step 3] Count the tallies for each distinct value and record totals
                in the third column of the table.
            \end{description}
        \subsection*{Relative-Frequency Distribution of Qualitative Data}
            A relative-frequency distribution of qualitative data is a listing of the
            distinct values and their relative frequencies. A relative-frequency
            distribution provides a table of the values of the observations and (relatively)
            how often they occur.
        \subsection*{Procedure to Construct a Relative-Frequency Distribution of Qualitative
        Data}
            \begin{description}
                \item[Step 1] Obtain a frequency distribution of the data.
                \item[Step 2] Divide each frequency by the total number of observation. 
            \end{description}
        \subsection*{Pie Chart}
            A pie chart is a disk divided into wedge-shaped pieces proportional to relative
            frequencies of the qualitative data.
        \subsection*{Pie Chart Procedure}
            \begin{description}
                \item[Step 1] Obtain a relative-frequency distribution of the data.
                \item[Step 2] Divide a disk into wedge-shaped pieces proportional to the
                relative frequencies.
                \item[Step 3] Label the slices with the distinct values and their relative
                frequencies.   
            \end{description}
        \subsection*{Bar Chart}
            A bar chart displays the distinct values of the qualitative data on a horizontal
            axis and the relative frequencies (or frequencies or percents) of those values
            on a vertical axis. The relative frequency of each distinct value is represented
            by a vertical axis. The relative frequency of each distinct value is represented
            by a vertical bar whose height is equal to the relative frequency of the value.
            The bars should be positioned so that they do not touch each other.
        \subsection*{Bar Chart Procedure}
            \begin{description}
                \item[Step 1] Obtain a relative-frequency distribution of the data.
                \item[Step 2] Draw a horizontal axis on which to place the bars and a
                vertical axis on which to display the relative frequencies.
                \item[Step 3] For each distinct value, construct a vertical bar whose height
                equals the relative frequency of that value.
                \item[Step 4] Label the bars with the distinct values, the horizontal axis
                with the name of the variable, and the vertical axis with "Relative 
                Frequency."  
            \end{description}

    \section*{Section 2.3}
    \noindent\rule{\textwidth}{0.4pt}
        \subsection*{Single-Value Grouping}
            In data sets with only a few discrete values the best grouping method is usually
            single-value grouping, where each value constitutes its own class.
        \subsection*{Limit Grouping}
            When the data set of integers has too many distinct values to be single-value 
            grouped, limit grouping may be the best method. Using class limits, values are
            grouped into sequential groups.
            \begin{description}
                \item[Lower class limits:] The smallest value that could go in a class.
                \item[Upper class limit:] The largest value that could go in a class. 
                \item[Class width:] The difference between the lower limit of a class and
                the lower limit of the next-higher class.
                \item[Class mark:] The mean of the upper and lower limits of a class.
            \end{description}
        \subsection*{Cutpoint Grouping}
            When the data set is continuous and its members are real valued the best
            grouping method may be cutpoint grouping. Using class cutpoints, values are
            grouped in sequential  groups.
            \begin{description}
                \item[Lower class cutpoint:] The smallest value that could go in a class.
                \item[Upper class cutpoint:] The smallest value that could go in the
                next-higher class (equivalent to the lower cutpoint of the next-higher
                class).
                \item[Class width:] The difference between the cutpoints of a class.
                \item[Class midpoint:] The mean of the two cutpoints of a class.
            \end{description}
        \subsection*{Histograms}
            A histogram provides a graph of the values of the observations and how often
            they occur. A histogram displays the classes of the quantitative data on a
            horizontal axis and the frequencies, relative frequencies, or percents of those
            classes on a vertical axis. The frequency, relative frequency, or percent of 
            each class is represented by a vertical bar whose height is equal to the 
            frequency (relative frequency, percent) of that class. The bars should be
            positioned so that they touch each other.
            \begin{itemize}
                \item For single-valued grouping, we use distinct values of observations
                to label the bars, with each such value centered under its bar.
                \item For limit grouping or cutpoint grouping, we use the lower class
                limits (or equivalently, lower class cutpoints) to label the bars. 
                \item Note: Some statisticians and technologies use class marks or class
                midpoints centered under the bars.
            \end{itemize}
            \subsubsection*{Histogram Procedure}
                \begin{description}
                    \item[Step 1] Obtain a frequency, relative-frequency, or percent
                    distribution of the data.
                    \item[Step 2] Draw a horizontal axis on which to place the bars and a
                    vertical axis on which to display the frequencies, relative 
                    frequencies, or percents.
                    \item[Step 3] For each class, construct a vertical bar whose height
                    equals the frequency, relative frequency, or percent of that class.
                    \item[Step 4] Label the bars with classes, the horizontal axis with the
                    name of the variable, and the vertical axis with "Frequency",
                    "Relative frequency," or "Percent".  
                \end{description}

        \subsection*{Dotplot}
            A dotplot is a graph in which each observation is plotted as a dot at an
            appropriate place above a horizontal axis. Observations having equal values
            are stacked vertically.
            \subsubsection*{Dotplot Procedure}
                \begin{description}
                    \item[Step 1] Draw a horizontal axis that displays the possible values
                    of the quantitative data.
                    \item[Step 2] Record each observation by placing a dot over the
                    appropriate value on the horizontal axis.
                    \item[Step 3] Label the horizontal axis with the name of the variable.
                \end{description}
        \subsection*{Stem-and -Leaf Diagram (Stemplot)}
            In a stem-and-leaf diagram (or stemplot), each observation is separated into
            two parts, namely;
            \begin{description}
                \item[Stem:] consists of all but the rightmost digit
                \item[Leaf;] the rightmost digit
            \end{description}
            \subsubsection*{Stem-and-Leaf Procedure}
                \begin{description}
                    \item[Step 1] Think of each observation as a stem, all but the
                    rightmost digit, and a leaf, the rightmost digit.
                    \item[Step 2] Write the stems from smallest to largest in a vertical
                    column to the left of a vertical rule.
                    \item[Step 3] Write each leaf to the right of the vertical rule in the
                    row that contains the appropriate stem.   
                    \item[Step 4] Arrange the leaves in each row in ascending order.
                \end{description}
            
    \section*{Section 2.4}
    \noindent\rule{\textwidth}{0.4pt}
        \subsection*{Distribution of a Data Set}
            The distribution of a data set is a table, graph, or formula that provides the
            values of the observations and how often they occur.
        \subsection*{Population and Sample Data}
            \begin{description}
                \item[Population data:] The values of a variable for the entire population.
                \item[Sample data:] The values of a variable for a sample of the population. 
            \end{description}
        \subsection*{Population and Sample Distribution of a Variable}
            The distribution of a population data is called the population distribution, or
            the distribution of the variable. The distribution of sample data is called
            sample distribution.
        \subsection*{Population and Sample Distributions}
            For a simple random sample, the sample distribution approximates the population
            distribution (i.e., the distribution of the variable under consideration). The
            larger the sample size, the better the approximation tends to be.
    
\end{document}