\documentclass[12pt]{article}
\usepackage[]{fullpage}
\title{Chapter 6: Notes}
\author{Evan Hunt}
\begin{document}
    \maketitle

    \noindent\rule{\textwidth}{0.4pt}
    \section[]{Section 6.1}
    \noindent\rule{\textwidth}{0.4pt}
        \subsection*{Basic Properties of Density Curves}
            \begin{description}
                \item[Property 1:] A density curve is always on or above the horizontal axis
                \item[Property 2:] The total area under a density curve (and above the horizontal
                axis) equals 1.                
            \end{description}
        \subsection*{Variables and Their Density Curves}
            For a variable with a density curve, the percentage of all possible observations of
            the variable that lie within any specified range equals (at least approximately) the
            corresponding area under the density curve, expressed as a percentage.
        \subsection*{Normally Distributed Variable}
            A variable is said to be a normally distributed variable or to have a normal
            distribution if its distribution has the shape of a normal curve.
        \subsection*{Normally Distributed Variables and the Normal-Curve Areas}
            For a normally distributed variable, the percentage of all possible observations that
            lie within any specified range equals the corresponding area under its associated
            normal curve,m expressed as a percentage. This result holds approximately for a
            variable that is approximately normally distributed.
        \subsection*{Standard Normal Distribution; Standard Normal Curve}
            A normally distributed variable having mean 0 and standard deviation 1 is said to
            have the standard normal distribution. Its associated normal curve is called the
            standard normal curve.
        \subsection*{Standardized Normally Distributed Variable}
            The standardized version of a normally distributed variable x,
            \begin{center}
                \[
                    z = \frac{x - \mu}{\sigma}, 
                \]
            \end{center}
            has the standard normal distribution. Subtracting from a normally distributed variable
            its mean and then dividing by its standard deviation results in a variable with the
            standard normal distribution.

    \section[]{Section 6.2}
    \noindent\rule{\textwidth}{0.4pt}
        \subsection*{Basic Properties of the Standard Normal Curve}
            \begin{description}
                \item[Property 1:] The total area under the standard normal curve is 1.
                \item[Property 2:] The standard normal curve extends indefinitely in both
                directions, approaching, but never touching, the horizontal axis.                 
                \item[Property 3:] The standard normal curve is symmetric about 0; that is, curve
                to the right of the mean is a mirror image of the curve to the left of the mean.
                \item[Property 4:] Almost all the area under the standard normal curve lies between
                -3 and 3.  
            \end{description}
        \subsection*{The $z_\alpha$ Notation}
            The symbol $z_\alpha$ is used to denote the z-score that has an area of $\alpha$ (alpha)
            to its right under the standard normal curve. Read "$z_\alpha$" as "z sub $\alpha$" or
            as "z $\alpha$".

    
    \section[]{Section 6.3}
    \noindent\rule{\textwidth}{0.4pt}
        \subsection*{Procedure to Determine a Percentage or Probability for a Normally Distributed
        Variable}
            \begin{description}
                \item[Step 1] Sketch the normal curve associated with the variable.
                \item[Step 2] Shade the region of interest and mark its delimiting $x$-value(s).
                \item[Step 3] Find the $z$-score(s) for the delimiting $x$ value(s) found in Step 2.
                \item[Step 4] Use a $z$table to find the area under the standard normal curve
                delimiter by the $z$-score(s) found in Step 3.               
            \end{description}
        \subsection*{Empirical Rule for Variables}
            For any variable whose distribution is bell-shaped (in particular, for any normally
            distributed variable), the following three properties hold.
            \begin{description}
                \item[Property 1:] Approximately 68\% of all possible observations lie within one
                standard deviation to either side of the mean, $[\mu - \sigma, \mu + \sigma]$.
                \item[Property 2:] Approximately 95\% of all possible observations lie within two
                standard deviations to either side of the mean, $[\mu - 2\sigma, \mu + 2\sigma]$. 
                \item[Property 3:] Approximately 99.7\% of all possible observations lie within
                three standard deviations to either side of the mean, 
                $[\mu - 3\sigma, \mu + 3\sigma]$.                
            \end{description}
        \subsection*{Procedure to Determine the Observations Corresponding to Specified
        percentage or Probability for a Normally Distributed Variable}
            \begin{description}
                \item[Step 1] Sketch the normal curve associated with the variable.
                \item[Step 2] Shade the region of interest.
                \item[Step 3] Use a $z$-table to determine the $z$-score(s) delimiting the region
                found in Step 2.
                \item[Step 4] Find the $x$-value(s) having the $z$-score(s) found in Step 3.
                \begin{center}
                    \[
                        x = \mu + z * \sigma
                    \]
                \end{center}
            \end{description}
        
    \section[]{Section 6.4}
    \noindent\rule{\textwidth}{0.4pt}
        \subsection*{Guidelines for Assessing Normality Using a Normal Probability Plot}
            A normal probability plot that falls nearly in a straight line indicates a normal
            variable, and one that does not indicates a nonnormal variable. To assess the
            normality of a variable using sample data, construct a normal probability plot.
            \begin{itemize}
                \item If the plot is roughly linear, you can assume that the variable is
                approximately normally distributed.
                \item If the plot is not roughly linear, you can assume that the variable is not
                approximately distributed.
            \end{itemize}
            These guidelines should be interpreted loosely for small samples but usually
            interpreted strictly for large samples.

    \section[]{Section 6.5}
    \noindent\rule{\textwidth}{0.4pt}
        \subsection*{}


\end{document}



 