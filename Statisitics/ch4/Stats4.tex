\documentclass[12pt]{article}
\usepackage[]{fullpage}
\title{Chapter 4: Probability}
\author{} 
\begin{document}
    \maketitle

    \noindent\rule{\textwidth}{0.4pt}
    \section*{Section 4.1}
    \noindent\rule{\textwidth}{0.4pt}
        \subsection*{Probability for Equally likely Outcomes (f/N Rule)}
            Suppose an experiment has N possible outcomes, all equally likely. An even that can
            occur in f ways has probability f/N of occurring:
            \begin{center}
                Probability of an event = $\frac{f}{N}$     
            \end{center}
            For an experiment with equally likely outcomes probabilities are identical to relative 
            frequencies (or percentages).
        \subsection*{Basic Properties of probabilities}
            \begin{description}
                \item[Property 1:] The probability of an event is always between 0 and 1, inclusive.
                \item[Property 2:] The probability of an event that cannot occur is 0. (An event 
                that cannot occur is called an impossible event.)
                \item[Property 3:] The probability of an event that must occur is 1. (An event that 
                must occur is called a certain event.)
            \end{description}

    \section*{Section 4.2}
    \noindent\rule{\textwidth}{0.4pt}
        \subsection*{Sample Space and Event}
            \begin{description}
                \item[Sample Space:] The collection of all possible outcomes for an experiment.
                \item[Event:] A collection of outcomes for the experiment, that is any subset of the
                sample space. An event occurs if only the outcome of the experiment is a member of
                the event.
            \end{description}
        \subsection*{Relationships Among Events}
            \begin{description}
                \item[(\(\neg A\)):] "The event "A does not occur"
                \item[(\(A \wedge B\)):] "The event "both A and B occur"
                \item[(\(A \vee B\):] "The event "either A or B occur"
            \end{description}
        \subsection*{Mutually Exclusive Events}
               Two or more events are mutually exclusive events if not two of them have outcomes in
               common.
               Events are mutually exclusive if no two of them can occur simultaneously or,
               equivalently, if at most one of the events can occur when the experiment is performed.

    \section*{Section 4.3}
    \noindent\rule{\textwidth}{0.4pt}
        \subsection*{Probability Notation}
            If E is an event, then P(E) represents the probability that event E occurs. It is read
            "the probability of E."
        \subsection*{The Special Addition Rule}
            If event A and event B are mutually exclusive, then
            \begin{center}
                \[
                    P(A \vee B) = P(A)  + P(B).  
                \]
            \end{center}
            More generally, if events A, B, C,$\dots$ are mutually exclusive, then
            \begin{center}
                \[
                    P(A \vee B \vee C \vee \dots) = P(A) + P(B) + P(C) + \dots    
                \]                
            \end{center}
            For mutually exclusive events, the probability that at least one occurs equals the sum
            of their individual probabilities.
        \subsection*{The Complementation Rule}
            For any event A,
            \begin{center}
                \[
                    P(A) = 1 - P(\neg A).                
                \]
            \end{center}
            The probability that an event occurs equals 1 minus the probability that it does not.
        \subsection*{The general Addition Rule}
            If A and B are any two events, then
            \begin{center}
                \[
                    P(A \vee B) = P(A) + P(B) - P(A \wedge B) 
                \]
            \end{center}
            For any two events, the probability that at least one occurs equals the sum of their
            individual probabilities minus the probabilities that both occur.

    \section*{Section 4.4}
    \noindent\rule{\textwidth}{0.4pt}
        \subsection*{Contingency Tables}
        \begin{description}
            \item[Univariate data:] data that comes from one variable of a population.
            \item[Bivariate data:] data that comes from two variables of a population.
            \item[Contingency table:] A frequency table for bivariate data.  
        \end{description}
                \begin{center}
                    \begin{table}[h!]
                        \centering
                        \caption{Contingency Table}
                        \begin{tabular}{l|l|l|l|l|l}
         & $A_1$               & $A_2$               & \dots & $A_n$               & $P(B_j)$          \\
\hline
$B_1$    & $P(A_1 \wedge B_1)$ & $P(A_2 \wedge B_1)$ & \dots & $P(A_n \wedge B_1)$ & $P(B_1)$         \\
$B_2$    & $P(A_1 \wedge B_2)$ & $P(A_2 \wedge B_2)$ & \dots & $P(A_n \wedge B_2)$ & $P(B_2)$         \\
\dots    & \dots               & \dots               & \dots & \dots               & \dots            \\
$B_m$    & $P(A_1 \wedge B_m)$ & $P(A_2 \wedge B_m)$ & \dots & $P(A_n \wedge B_m)$ & $P(B_m)$         \\
$P(A_i)$ & $P(A_1)$            & $P(A_2)$            & \dots & $P(A_n)$            & $P(A \vee B) =1$ \\
                        \end{tabular}
                      \end{table}
                \end{center}
    \section*{Section 4.5}
        \subsection*{Conditional Probability}
            \subsubsection*{}
                The probability that event B occurs given that event A occurs is called conditional 
                probability. It is denoted P(B|A),
                which is read "the probability of B given A." We call A the given event.
            \subsubsection*{}
                A conditional probability of an event is the probability that the event occurs under
                the assumption that another event occurs.
        \subsection*{The conditional Probability Rule}
            If A and B are any two events with P(A) > 0, then
            \begin{center}
                \[
                P(B|A) = \frac{P(A \wedge B)}{P(A)}
                \]
            \end{center}
            The conditional probability of one event given another equals the probability that both 
            events occur divided by the probability of the given event.
            
    \section*{Section 4.6}
    \noindent\rule{\textwidth}{0.4pt}
        \subsection*{The General Multiplication Rule}
            If A and B are any two events, then 
            \begin{center}
                \[
                    P(A \wedge B) = P(A) * P(B|A)
                \]
            \end{center}
            For any two events, the probability that both occur equals the probability that a specified
            once occurs tines the conditional probability of the other event, given the specified event.         
        \subsection*{Independent Events}
            Event B is said to be independent of event A if P(B|A) = P(B). One event is independent of 
            another event if knowing whether the latter event occurs
            does not affect the probability of the former event.
        \subsection*{The Special Multiplication Rule (for Two Independent Events)}
            If A and B are independent events, then
            \begin{center}
                \[
                    P(A \wedge B) = P(A) * P(B)    
                \]
                and conversely, if $P(A \wedge B) = P(A) * P(B)$, then A and B are independent events. 
            \end{center}
            Two events are independent if and only if the probability that both occur equals the product
            of their individual probabilities.
        \subsection*{The Special Multiplication Rule}
            If events A, B, C,$\dots$ are independent, then
            \begin{center}
                \[
                    P(A \wedge B \wedge C \wedge \dots) = P(A) * P(B) * P(C) \dots     
                \]
            \end{center}
            For independent events,the probability that they all occur equals the product of their 
            individual probabilities.

    \section*{Section 4.7}
        \subsection*{The Rule of total Probability}
        Suppose that events $A_1, A_2,\dots, A_k$ are mutually exclusive and exhaustive; that is, 
        exactly one of them must occur. Then for any event B,
        \begin{center}
            \[
                P(B) = \sum_{i=1}^n P(A_i) * P(B|A_i)
            \]
        \end{center}
        \subsection*{Bayes's Rule}
            Suppose that events $A_1, A_2,..., A_k$ are mutually exclusive and exhaustive. Then for 
            any event B,
            \begin{center}
                \[
                    P(A_i|B) = \frac{P(A_i) * P(B|A_i)}{\sum_[j=1]^k P(A_j) * P(B|A_j)}    
                \]
            \end{center}
            where $A_i$ can be any one of events $A_1, A_2,\dots, A_k$.
            
    \section*{Section 4.8}
        \subsection*{The Basic Counting Rule (BCR)'}
            Suppose that r actions are to be performed in a definite order. Further suppose that
            there are $m_1$ possibility for the first action and that corresponding to each of
            these possibilities are $m_1 * m_2 * \dots * m_r$ possibilities altogether for the r
            actions. In other words, the total
            number of ways that several actions can occur equals the product of the individual 
            number of ways for each action.
        \subsection*{Factorials}
            The product of the first $k$ positive integers (counting numbers) is call k factorial
            and is denoted k!. In symbols,
            \begin{center}
                \[
                    k! = k * (k-1) * \dots * 2 * 1
                \]
            \end{center}
            We also define 0! = 1.
        \subsection*{The Permutations Rule}
            The number of possible permutations of $k$ objects from a collection of $n$ objects 
            is given the formula
            \begin{center}
                \[
                    _nP_k = \frac{n!}{(n-k)!}
                \]
            \end{center}
        \subsection*{The Special Permutation Rule}
            The number of permutations of $n$ objects among themselves is n!.
        \subsection*{The Combinations Rule}
            The number of possible combinations of $k$ objects from a collection of $n$ objects 
            is given by the formula
            \begin{center}
                \[
                    _nC_k = \frac{n!}{k!(n-k)!}    
                \]                
            \end{center}
        \subsection*{Number of Possible Samples}
            The number of possible samples of size $n$ from a population of size $N$ is $_NC_n$.
\end{document}
