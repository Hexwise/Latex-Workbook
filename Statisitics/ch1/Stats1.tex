\documentclass[12pt]{article}
\usepackage[]{fullpage}
\title{Chapter 1: Notes}
\author{}
\begin{document}
    \maketitle

    \noindent\rule{\textwidth}{0.4pt}
    \section*{Section 1.1}
    \noindent\rule{\textwidth}{0.4pt}
        \subsection*{Descriptive Statistics}
            Descriptive statistics consists of methods for organizing and summarizing
            information.
        \subsection*{Population and Sample}
            \begin{description}
                \item[Population:] The collection of all individuals or items under
                consideration in a statistical study. 
                \item[Sample:] That part of the population from which information obtained 
            \end{description}
        \subsection*{Inferential Statistics}
            Inferential statistics consists of methods of drawing and measuring the
            reliability of conclusions about a population based on information obtained
            from a sample of the population.

    \section*{Section 1.2}
    \noindent\rule{\textwidth}{0.4pt}
        \subsection*{Simple Random Sampling; Simple Random Sample}
            \begin{description}
                \item[Simple random sampling:] A sampling procedure for which each
                possible sample of given size is equally likely to be the one obtained. 
                \item[Simple random sample:] A sample obtained by simple random sampling.
            \end{description}
        Simple random sampling corresponds to our intuitive notion of random selection
        by lot.

    \section*{Section 1.3}
    \noindent\rule{\textwidth}{0.4pt}
        \subsection*{Systematic Random Sampling}
            \begin{description}
                \item[Step 1] Divide the population size by the sample size and round the
                result down to the nearest whole number, $m$. 
                \item[Step 2] Use a random-number table or technology to obtain a number,
                $k$, between 1 and $m$.
                \item[Step 3] Select for the sample those members of the population that
                are numbered $k, k+m, k+2m,\dots$
            \end{description}
        \subsection*{Cluster Sampling}
            \begin{description}
                \item[Step 1] Divide the population into groups (clusters).
                \item[Step 2] Obtain a simple random sample of the clusters.
                \item[Step 3] Use all the members of the clusters obtained in Step 2 as
                the sample.   
            \end{description}
        \subsection*{Stratified Random Sampling with Proportional Allocation}
            \begin{description}
                \item[Step 1] Divide the population into subpopulations (strata).
                \item[Step 2] From each stratum, obtain a simple random sample of size
                proportional to the size of the stratum; that is, the sample size for a
                stratum equals the total sample size times the stratum size divided by the
                population size.
                \item[Step 3] Use all the members obtained in Step 2 as the sample.  
            \end{description}

    \section*{Section 1.4}
    \noindent\rule{\textwidth}{0.4pt}
        \subsection*{Experimental Units; Subjects}
            In a designed experiment, the individuals or items on which the experiment
            is performed are called experimental units. When the experimental units are
            humans, the rem subject is often used in place of experimental unit.
        \subsection*{Principles of Experimental Design}
            The following principles of experimental design enable a researcher to
            conclude that differences in the results of an experiment not reasonably
            attributable to chance are likely caused by the treatments.
            \begin{description}
                \item[Control:] Two or more treatments should be compared.
                \item[Randomization:] The experimental units should be randomly divided
                into groups to avoid unintentional selection bias in constituting the
                groups.
                \item[Replication:] A sufficient number of experimental units should be
                used to ensure that randomization creates groups that resemble each other
                closely and to increase the chances of detecting any differences among the
                treatments.   
            \end{description}
        \subsection*{Response Variable, Factors, Levels, and Treatments}
            \begin{description}
                \item[Response variable:] The characteristic of the experimental outcomes
                that is to be measured or observed.
                \item[Factor:] A variable whose effect on the response variable is of
                interest in the experiment.
                \item[Levels:] The possible values of a factor  
                \item[Treatment:] Each experimental condition. For one-factor experiments,
                the treatments are the levels of the single factor. For multifactor
                experiments, each treatment is a combination of levels of the factors.
            \end{description}
        \subsection*{Completely Randomized Design}
            In a completely randomized Design, all the experimental units are assigned
            randomly among all the treatments.
        \subsection*{Randomized Block Design}
            In a randomized block design, the experimental units are assigned randomly
            among all the treatments separately within each block.

\end{document}



