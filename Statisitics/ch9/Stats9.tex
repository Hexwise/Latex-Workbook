\documentclass[12pt]{article}
\usepackage[]{fullpage}
\title{Chapter 9: Hypothesis Tests for One Population Mean}
\author{}
\begin{document}
    \maketitle

    \noindent\rule{\textwidth}{0.4pt}
    \section*{Section 9.1}
    \noindent\rule{\textwidth}{0.4pt}
        \subsection*{Null and Alternative Hypotheses; Hypothesis Test}
            \begin{description}
                \item[Null hypothesis:] A hypothesis to be tested. We use the symbol $H_0$ to
                represent the null hypothesis.
                \item[Alternative hypothesis:] A hypothesis to be considered as an alternative
                to the null hypothesis. We use the symbol $H_a$ to represent the
                alternative hypothesis.
                \item[Hypothesis test:] The problem in a hypothesis test is to decide whether
                the null hypothesis should be rejected.                
            \end{description}
            \subsubsection*{Basic Logic of Hypothesis Testing}
                Take a random sample from the population. If the sample data are consistent
                with the null hypothesis, do not reject the null hypothesis; if the sample data
                are inconsistent with the null Hypothesis and supportive of the alternative
                hypothesis, reject the null hypothesis in favor of the alternative hypothesis.
        \subsection*{Type I and Type II Errors}
            \begin{description}
                \item[Type I error:] Rejecting the null hypothesis when it is in fact true.
                \item[Type II error:] Not rejecting the null hypothesis when it is in fact
                false. 
            \end{description}    
            \subsubsection*{Significance Level}
                The probability of making a Type I error, that is, of rejecting the null
                hypothesis, is called the significance level, $\alpha$, of a hypothesis test.
            \subsubsection*{Relation between Type I and Type II Error Probabilities}
                For a fixed sample size, the smaller we specify the significance level,
                $\alpha$, the larger will be the probability, $\beta$, of not rejecting a 
                false hypothesis.
        \subsection*{Possible Conclusions for a Hypothesis Test}
            Suppose that a hypothesis test is conducted at a small significance level.
            \begin{itemize}
                \item If the hypothesis is rejected, we could conclude that the data provided
                sufficient evidence to support the alternative hypothesis.
                \item If the null hypothesis is not rejected, we conclude that the data do not
                provide sufficient evidence to support the alternative hypothesis.                 
            \end{itemize}

    \section*{Section 9.2}
    \noindent\rule{\textwidth}{0.4pt}
        \subsection*{Rejection Region, Nonrejection Region, and Critical Values}
            \begin{description}
                \item[Rejection region:] The set of values for the test statistic that leads
                to nonrejection of the null hypothesis.
                \item[Nonrejection region:] The set of values for the test statistic that leads
                to nonrejection of the null hypothesis.
                \item[Critical Value(s):] The value or values of the test statistic that
                separate the rejection and nonrejection regions. A critical value is considered
                part of the rejection region.  
            \end{description}
            If the value of the test statistic falls in the rejection region, reject the null
            hypothesis; otherwise, do not reject the null hypothesis.
            \subsubsection*{Obtaining Critical Values}
                Suppose that a hypothesis test is to be performed at the significance level
                $\alpha$. Then the critical value(s) must be chosen so that, if the null
                hypothesis is true, the probability is $\alpha$ that the test statistic will
                fall in the rejection region.
        \subsection*{Critical-Value Approach to Hypothesis Testing}
            \begin{description}
                \item[Step 1] State the null and alternative hypothesis.
                \item[Step 2] Decide on the significance level, $\alpha$.
                \item[Step 3] Compute the value of the test statistic.
                \item[Step 4] Determine the critical value(s).
                \item[Step 5] If the value of the test statistic falls in the rejection
                region, reject $H_0$; otherwise, do not reject $H_0$.
                \item[Step 6] Interpret the result of the hypothesis test.                
            \end{description}

    \section*{Section 9.3}
    \noindent\rule{\textwidth}{0.4pt}
        \subsection*{$P$-Value}
            The $P$-value of a hypothesis test is the the probability of getting sample data
            at least as inconsistent with the null hypothesis (and supportive of the 
            alternative) hypothesis) as the sample data actually obtained. We use the letter
            $P$ to denote the $P$-value. Small $P$-values provide evidence against the null
            hypothesis; large $P$-values do not.
        \subsection*{Decision Criterion for Hypothesis Test Using the $P$-Values}
            If the $P$-value is less than or equal to the specified significance level, reject
            the null hypothesis; otherwise, do not reject the null hypothesis. In other words,
            if $P \leq \alpha$, reject $H_0$; otherwise, do not reject $H_\alpha$.
        \subsection*{$P$-Value as the Observed Significance Level}
            The $P$-value of a hypothesis test equals the smallest significance level at which
            the null hypothesis can be rejected, that is, the smallest significance level for
            which the observed sample data result in rejection of $H_0$.
        \subsection*{Determining a $P$-Value}
            To determine the P-Value of a hypothesis test, we assume that the null hypothesis
            is true and compute the probability of observing a value of the test statistic as
            extreme as or more extreme that that observed. By extreme we mean "far from what
            we would expect to observe if the null hypothesis is true."
        \subsection*{$P$-Value Approach to Hypothesis Testing}
            \begin{description}
                \item[Step 1] State the null and alternative hypotheses.
                \item[Step 2] Decide on the significance level, $\alpha$. 
                \item[Step 3] Compute the value of the test statistic.
                \item[Step 4] Determine the $P$-value, $P$.
                \item[Step 5] If $P \leq \alpha$, reject $H_a$; otherwise, do not reject $H_0$.
                \item[Step 6] Interpret the result of the hypothesis test.    
            \end{description}
            \subsubsection*{Hypothesis Tests Without Significance Levels:} Many researchers do
            do not explicitly refer to significance levels. Instead, they simply obtain the
            $P$-value and use it (or let the reader us it) to assess the strength of the
            evidence against the null hypothesis.
    
    \section*{Section 9.4}
    \noindent\rule{\textwidth}{0.4pt}
        \subsection*{One-Mean z-Test}
            \begin{description}
                \item[Purpose] To perform a hypothesis test for a population mean, $\mu$.
                \item[Assumptions]
                \begin{enumerate}
                    \item Simple random sample.
                    \item Normal population or large sample.
                    \item $\sigma$ known
                \end{enumerate}  
                \item[Step 1] The null hypothesis is $H_0$: $\mu = \mu_0$, and the alternative
                hypothesis is
                \begin{center}
                    \[
                        H_a: \mu != \mu_0 \quad or \quad H_a: \mu < \mu_0 \quad or \quad 
                        H_a: \mu > \mu_0    
                    \]
                    \[
                        (Two \quad tailed)\quad\quad\quad(Left \quad Tailed)\quad\quad\quad
                        (Right \quad tailed)    
                    \]                    
                \end{center}
                \item[Step 2] Decide on the significance level, $\alpha$.
                \item[Step 3] Compute the value of the test statistic 
                \begin{center}
                    \[
                        z = \frac{\bar{x} - \mu_0}{\sigma / \sqrt{n}}    
                    \]
                \end{center}
                and denote that value $z_0$.
            \end{description}
            \subsubsection*{Critical-Value Approach}
                \begin{description}
                    \item[Step 4] The critical value(s) are
                    \begin{center}
                        \[
                            \pm z _{\alpha/2} \quad\quad or \quad\quad -z_{\alpha} \quad\quad or
                            \quad\quad z_{\alpha}     
                        \]
                        \[
                            (Two \quad tailed)\quad\quad\quad(Left \quad tailed)\quad\quad\quad
                            (Right \quad tailed)
                        \]
                    \end{center} 
                    Use a $z$-table to find the critical values.
                    \item[Step 5] If the value of the test statistic fall in the rejection
                    region, reject $H_0$; otherwise, do not reject $H_0$.
                    \item[Step 6] Interpret the results of the hypothesis test. 
                \end{description}
            \subsubsection*{$P$-Value Approach}
                \begin{description}
                    \item[Step 4] Use a $z$-table to obtain the $P$-value.
                    \item[Step 5] If $P \leq \alpha$, reject $H_0$; otherwise, do not reject
                    $H_0$.
                    \item[Step 6] Interpret the results of the hypothesis test.
                    \item[Note:] The hypothesis test is exact for normal populations and is
                    approximate for large samples from nonnormal populations.  
                \end{description}
        \subsection*{When to Use the One-Mean $z$-Test}
            \begin{itemize}
                \item For small samples, less than 15 roughly, the $z$-test should be used
                only when the variable under consideration is normally distributed or very
                close to being so.
                \item For samples of moderate size, between 15 and 30 roughly, the $z$-test
                can be used unless the data contain outliers or the variable under
                consideration is far from being normally distributed.
                \item For large samples, around 30 or more, the $z$-test can be used
                essentially without restriction. However, if outliers are present and their
                removal is not justified , you should perform the hypothesis test once with
                outliers and once without them to see what effect the outliers have. If the
                conclusion is affected, use a different procedure or take another sample,
                if possible.
                \item If outliers are present but their removal is justified and result in
                a data set for which the $z$-test us appropriate (as previously stated),
                the procedure can be used.                
            \end{itemize}

    \section*{Section 9.5}
    \noindent\rule{\textwidth}{0.4pt}
        \subsection*{One-Mean $t$-Test}
            \begin{description}
                \item[Purpose] To perform a hypothesis test for a population mean, $\mu$.
                \item[Assumptions]
                \begin{enumerate}
                    \item Simple random sample
                    \item Normal Population or large sample
                    \item $\sigma$ unknown 
                \end{enumerate} 
                \item[Step 1] The null hypothesis is $H_0$: $\mu = \mu_0$, and the
                alternative hypothesis is 
                \begin{center}
                    \[
                        H_0: \mu != \mu_0 \quad or \quad H_0: \mu < \mu_0 \quad or \quad
                        H_0: \mu > \mu_0    
                    \]
                    \[
                        (Two \quad tailed)\quad\quad\quad(Left \quad tailed)\quad\quad\quad
                        (Right \quad tailed)    
                    \]
                \end{center}
                \item[Step 2] Decide on the significance level, $\alpha$.
                \item[Step 3] Compute the value of the test statistic
                \begin{center}   
                    \[
                        t = \frac{\bar{x} - \mu_0}{s/\sqrt{n}}    
                    \]
                \end{center}
                and denote that value $t_0$.
            \end{description}
            \subsubsection*{Critical-Value Approach}
                \begin{description}
                    \item[Step 4] The critical value(s) are
                    \begin{center}
                        \[
                            \pm t_{\alpha/2} \quad\quad or \quad\quad -t_{\alpha} \quad\quad or
                            \quad\quad t_{\alpha}    
                        \]
                        \[
                            (Two \quad tailed)\quad\quad\quad(Left \quad tailed)\quad\quad\quad
                            (Right \quad tailed)
                        \]
                    \end{center} 
                    \item[Step 5] If the value of the test statistic fall in the rejection
                    region, reject $H_0$; otherwise, do not reject $H_0$. 
                    \item[Step 6] Interpret the results of the hypothesis test.
                \end{description}
            \subsubsection*{$P$-Value Approach}
                \begin{description}
                    \item[Step 4] The $t$ statistic had $df = n-1$. Use a $t$-table to
                    estimate the $P$-value; or obtain it exactly by using technology.
                    \item[Step 5] If $P \leq \alpha$, reject $H_0$; otherwise, do not
                    reject $H_0$.  
                    \item[Step 6] Interpret the results of the hypothesis test.
                \end{description}

    \section*{Section 9.6}
    \noindent\rule{\textwidth}{0.4pt}


    \section*{Section 9.7}
    \noindent\rule{\textwidth}{0.4pt}


    \section*{Section 9.8}
    \noindent\rule{\textwidth}{0.4pt}


\end{document}



