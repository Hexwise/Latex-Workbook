\documentclass[12pt]{article}
\usepackage[]{fullpage}
\title{Chapter 12: Notes}
\author{Evan Hunt}
\begin{document}
    \maketitle

    \noindent\rule{\textwidth}{0.4pt}
    \section*{Section 12.1}
    \noindent\rule{\textwidth}{0.4pt}
        \subsection*{Population Proportion and Sample Proportion}
            Consider a population in which each member either has or does not have a specified
            attribute. Then we use the following notation and terminology.
            \begin{description}
                \item[Population proportion, p:] The proportion (percentage) of the entire
                population that has the specified attribute.
                \item[Sample proportion, $\hat{p}$:] The proportion (percentage) of a sample
                from the population that has the specified attribute. 
            \end{description}
        \subsection*{Sample Population}
            A sample proportion, $\hat{p}$, is computed by using the formula
            \begin{center}
                \[
                    \hat{p} = \frac{x}{n},    
                \]
            \end{center}
            where $x$ denotes the number of members in the sample that have the specified
            attribute and, as usual, $n$ denotes the sample size.
            \subsubsection*{Note:}
                For convenience, we sometimes refer to $x$ (the number of members in the sample
                that have the specified attribute) as the number of successes and to $n - x$
                (the number of member in the sample that do not have the specified attribute)
                as the number of failures. In this context, the words success and failure may
                not have their ordinary meanings.
        \subsection*{The sampling Distribution of the Sample Proportion`}
            For samples of size $n$,
            \begin{itemize}
                \item the mean of $\hat{p}$ equals the population proportion: $\mu_{\hat{p}}
                = p$ (i.e., the sample proportion is an unbiased estimator of the population
                proportion);
                \item the standard deviation of $\hat{p}$ equals the square root of the
                product of the population proportion and one mine the population proportion
                divided by sample size: $\sigma_{\hat{p} = \sqrt{p(1-p)/n}}$; and
                \item $\hat{p}$ is approximately normally distributed for large $n$.
            \end{itemize}
            If $n$ is large, the possible samples of size $n$ have approximately a normal
            distribution with mean $p$ and standard deviation $\sqrt{p(1-p)/n}$.
        \subsection{One-Proportion $z$-Interval Procedure}
            \begin{description}
                \item[Purpose] To find confidence interval for a population proportion, $p$
                \item[Assumptions] 
                \begin{enumerate}
                    \item Simple random sample
                    \item The number of successes, $x$, and the number of failures, $n-x$,
                    are both 5 or greater.                    
                \end{enumerate}
                \item[Step 1] For a confidence level of $1-\alpha$, use a $z$-table to find
                $z_{\alpha/2}$.
                \item[Step 2] The confidence interval for $p$ is from 
                \begin{center}
                    \[
                        \hat{p}-z_{\alpha/2}*\sqrt{\hat{p}(1-\hat{p})/n} to
                        \hat{p}+z_{\alpha/2}*\sqrt{\hat{p}(1-\hat{p})/n},
                    \]
                \end{center}
                where $z_{\alpha/2}$ is found in Step 1, $n$ is the sample size, and 
                $\hat{p}=x/n$ is the sample proportion. 
                \item[Step 3] Interpret the confidence interval.
            \end{description}
        \subsection*{Margin of Error for the Estimate of $p$}
            The margin of error for the estimate of $p$ is
            \begin{center}
                \[
                    E = z_{\alpha/2}*\sqrt{\hat{p}(1-\hat{p})/n}.
                \]
            \end{center}
            The margin of error for the estimate of a population proportion indicates the
            accuracy with which a sample proportion estimates the unknown population
            proportion at the specified confidence level.
        \subsection*{Sample Size for Estimating $p$}
            \begin{itemize}
                \item A $(1-\alpha)$-level confidence interval for population proportion
                that has a margin of error of at most $E$ can be obtained by choosing
                \begin{center}
                    \[
                        n = 0.25(z_{\alpha/2}/E)^2    
                    \]
                \end{center}
                rounded up to the nearest whole number.
                \item If you can make an educated guess, $\hat{p}_g$ (g for guess), for
                the observed value of $\hat{p}$, then you should instead choose
                \begin{center}
                    \[
                        n = \hat{p}_g(1-\hat{p}_g)(z_{\alpha/2}/E)^2
                    \]
                \end{center}
                rounded up to the nearest whole number. 
                \item If you have in mind a likely range for the observed value of 
                $\hat{p}$, then you should apply the preceding formula with you educated
                guess for observed value of $\hat{p}$ being the value in range closest to
                0.5.
            \end{itemize}

    \section*{Section 12.2}
    \noindent\rule{\textwidth}{0.4pt}
        \subsection*{One-Proportion $z$-Test}
            \begin{description}
                \item[Purpose] To perform a hypothesis test for a population proportion,
                $p$.
                \item[Assumption] 
                \begin{enumerate}
                    \item Simple random sample
                    \item Both $np_0$ and $n(1-p0)$ are 5 or greater
                \end{enumerate}
                \item[Step 1] The null hypothesis is $H_0$: $p = p_0$, and the alternative
                hypothesis is 
                \begin{center}
                    \[
                        H_a: p != p_0 \quad or \quad H_a: p < p_0 \quad or \quad 
                        H_a: p > p_0    
                    \]
                    \[
                        (Two \quad tailed)\quad\quad\quad(Left \quad Tailed)\quad\quad\quad
                        (Right \quad tailed)    
                    \]                    
                \end{center}
                \item[Step 2] Decide the significance level, $\alpha$.
                \item[Step 3] Compute the value of the test statistic
                \begin{center}
                    \[
                        z = \frac{\hat{p}-p_0}{\sqrt{p_0(1-p_0)/n}}    
                    \]
                \end{center}  
                and denote that value $z_0$.
            \end{description}
            \subsubsection*{Critical-Value Approach}
                \begin{description}
                    \item[Step 4] The critical value(s) are
                    \begin{center}
                        \[
                            \pm z _{\alpha/2} \quad\quad or \quad\quad -z_{\alpha} \quad
                            \quad or \quad\quad z_{\alpha}     
                        \]
                        \[
                            (Two \quad tailed)\quad\quad\quad(Left \quad tailed)\quad\quad
                            \quad(Right \quad tailed)
                        \]
                    \end{center}
                    \item[Step 5] If the value of the test statistic falls in the rejection
                    region, reject $H_0$: otherwise, do not reject $H_0$.
                    \item[Step 6] Interpret the results of the hypothesis test.
                \end{description}
            \subsubsection*{$P$-Value Approach}
                \begin{description}
                    \item[Step 4] Use a $z$-table to obtain the $P$-value.
                    \item[Step 5] If $P \leq \alpha$, reject $H_0$: otherwise, do not reject
                    $H_0$.
                    \item[Step 6] Interpret the results of the hypothesis test.
                \end{description}

\end{document}



