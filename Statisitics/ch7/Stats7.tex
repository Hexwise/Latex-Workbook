\documentclass[12pt]{article}
\usepackage[]{fullpage}
\title{Chapter 7: Notes}
\author{}
\begin{document}
    \maketitle

    \noindent\rule{\textwidth}{0.4pt}
    \section*{Section 7.1}
    \noindent\rule{\textwidth}{0.4pt}
        \subsection*{Sampling Error}    
            Sampling error is the error resulting from using a sample to estimate a population
            characteristic.
        \subsection*{Sampling Distribution of the Sample Mean}
            For a variable $x$ and a given sample size, the distribution of the variable
            $\bar{x}$ is called the sampling distribution of the sample mean. The sampling 
            distribution of the sample mean is the distribution of all possible sample means
            for the samples of a given size.
        \subsection*{Sample Size and Sampling Error}
            The larger the sample size, the smaller the sampling error tends to be in estimating
            a population mean, $\mu$, by a sample mean. $\bar{x}$. The possible sample means
            cluster more closely around the population mean as the sample size increases.

    \section*{Section 7.2}
    \noindent\rule{\textwidth}{0.4pt}
        \subsection*{Mean of the Sample Mean}
            For samples of size $n$, the mean of the variable $\bar{x}$ equals the mean of the
            variable under consideration. In symbols,
            \begin{center}
                \[
                    \mu_{\bar{x}} = \mu    
                \]
            \end{center}
            For any sample size, the mean of all possible samples equals the population mean.
        \subsection*{Standard Deviation of the Sample Mean}
            For samples of size $n$, the standard deviation of the variable $\bar{x}$ equals the 
            standard deviation of the variable under consideration divided by the square roots
            of the sample size. In symbols,
            \begin{center}
                \[
                    \sigma_{\bar{x}} = \frac{\sigma}{\sqrt{n}}
                \]
            \end{center}
            For each sample size, the standard deviation of all possible sample means equals the
            standard deviation divided bu the square root of the sample size.
            \subsubsection*{Note:}
                In the formula for the standard deviation of $\bar{x}$, the sample size, $n$
                appears in the denominator. this explains mathematically why the standard
                deviation of $\bar{x}$ decreases as the sample size increases.
    \section*{Section 7.3}
    \noindent\rule{\textwidth}{0.4pt}
        \subsection*{Sampling Distribution of the Sample Mean for a Normally Distributed
        Variable}
            For a normally distributed variables, the possible sample means for samples of given
            size are also normally distributed. Suppose that a variable $x$ of a population is
            normally distributed with mean $\mu$ and standard deviation $\sigma$. Then, for
            samples of size $n$, the variable $\bar{x}$ is also normally distributed and has mean
            $\mu$ and standard deviation $\sigma / sqrt{n}$.
        \subsection*{The Central Limit Theorem (CLT)}
            For a relatively large sample size, the variable $\bar{x}$ is approximately normally
            distributed, regardless of the distribution of the variable under consideration. The
            approximation becomes better with increasing sample size.
        \subsection*{Sampling Distribution of the Sample Mean}
            Suppose that a variable $x$ of a population has mean $\mu$ and standard deviation
            $\sigma$. Then, for samples of size $n$.
            \begin{itemize}
                \item the mean of $\bar{x}$ equals the population mean, or $\mu_{\bar{x}} = \mu$;                
                \item the standard deviation of $\bar{x}$ equals the population standard
                deviation divided by the square roo of the sample size, or $\sigma_{bar{x}} =
                \frac{\sigma}{\sqrt{n}}$;
                \item if $x$ is normally distributed, so is $\bar{x}$, regardless of sample size;
                and
                \item if the sample size large, $\bar{x}$ is approximately normally distributed,
                regardless of the distribution of $x$.
            \end{itemize}
            If a variable is normally distributed or the sample size large, then the possible 
            sample means have, at least approximately, a normal distribution with mean $\mu$
            and standard deviation $\sigma / sqrt{n}$.  

\end{document}