\documentclass[12pt]{article}
\usepackage[]{fullpage}
\title{Chapter 10: Notes}
\author{Evan Hunt}
\begin{document}
    \maketitle

    \noindent\rule{\textwidth}{0.4pt}
    \section*{Section 10.1}
    \noindent\rule{\textwidth}{0.4pt}
        \subsection*[The sampling Distribution of the Difference between Two Sample Means
        for Independent Samples]
            Suppose that $x$ is a normally distributed variable on each of two populations.
            Then, for independent samples of size $n_1$ and $n_2$ from two populations,
            \begin{itemize}
                \item $\mu_{\bar{x_1}-\bar{x_2}} = \mu_1 - \mu_2$,
                \item $\sigma_{\bar{x_1}-\bar{x_2}} = \sqrt{(\sigma_1^2/n_1) +
                (\sigma_2^2/n_2)}$, and
                \item $\bar{x_1} - \bar{x_2}$ is normally distributed.                
            \end{itemize}
            
    \section*{Section 10.2}
    \noindent\rule{\textwidth}{0.4pt}
            \subsection*{Distribution of the Pooled $t$-Statistic}
                Suppose that $x$ is normally distriuted variable on each of two
                populations and that the population standard deviations are equal. Then,
                for independent samples od sizes $n_1$ and $n_2$ from the two populations,
                the variable
                \begin{center}
                    \[
                        t = \frac{(\bar{x_1}-\bar{x_2})-(\mu_1-\mu_2)}{s_p\sqrt{(1/n_1)+(1/n_2)}}    
                    \]
                has the $t$-Distribution with $df = n_1 + n_2 - 2$.
                \end{center}
            \subsection*{Pooled $t$ Test}
                \begin{description}
                    \item[Purpose] To perform a hypothesis test to compare two population
                    means, $\mu_1$ and $\mu_2$.
                    \item[Assumptions] 
                    \begin{enumerate}
                        \item Simple random samples
                        \item Independent samples
                        \item Normal populations or large samples
                        \item Equal population standard deviation, $\sigma$
                    \end{enumerate}
                    \item[Step 1] The null hypothesis is $H_0$: $\mu_1 = \mu_2$, and the
                    alternative hypothesis is
                    \begin{center}
                        \[
                            H_a: \mu != \mu_0 \quad or \quad H_a: \mu < \mu_0 \quad or \quad 
                            H_a: \mu > \mu_0    
                        \]
                        \[
                            (Two \quad tailed)\quad\quad\quad(Left \quad Tailed)\quad\quad\quad
                            (Right \quad tailed)    
                        \]                    
                    \end{center}
                    \item[Step 2] Decide on the significance level, $\alpha$.
                    \item[Step 3] Compute the value of the test statistic
                    \begin{center}
                        \[
                            t = \frac{\bar{x_1} - \bar{x_2}}{s_p\sqrt{(1/n_2) + (1/n_2)}}, 
                        \]
                    \end{center}
                    where
                    \begin{center}
                       \[
                            s_p = \sqrt{\frac{(n_1 - 1)s_1^2 + (n_2 -1)s_2^2}{n_1 + n_2 - 2}}.
                       \]    
                    \end{center}
                    Denote the value of the test statistic $t_0$.
                \end{description}
                \subsubsection*{Critical-Value Approach}
                    \begin{description}
                        \item[Step 4] The critical value(s) are
                        \begin{center}
                            \[
                                \pm z _{\alpha/2} \quad\quad or \quad\quad -z_{\alpha} \quad
                                \quad or \quad\quad z_{\alpha}     
                            \]
                            \[
                                (Two \quad tailed)\quad\quad\quad(Left \quad tailed)\quad\quad
                                \quad(Right \quad tailed)
                            \]
                        \end{center}
                        \item[Step 5] If the value of the test statistic falls in the rejection
                        region, reject $H_0$; otherwise, do not reject $H_0$.
                        \item[Step 6] Interpret the results of the hypothesis test.  
                    \end{description}
            \subsection*{$P$-Value Approach}
                \begin{description}
                    \item[Step 4] The $t$-statistic has $df = n_1 + n_2 - 2$. Use a $t$-table
                    to estimate the $P$-value, or obtain it exactly using technology.
                    \item[Step 5] If $P \leq \alpha$, reject $H_0$; otherwise, do not reject  
                    $H_0$.
                    \item[Step 6] Interpret the results of the hypothesis test.
                    \item[Note:] The hypothesis test is exact for normal populations and is
                    approximately correct for large samples from nonnormal populations.   
                \end{description}
            \subsection*{Pooled $t$-Interval Procedure}
                \begin{description}
                    \item[Purpose] To find a confidence interval for the difference between
                    two population means, $\mu_1$ and $\mu_2$. 
                    \item[Assumptions]
                    \begin{enumerate}
                        \item Simple random samples
                        \item Independent samples
                        \item Normal populations or large samples
                        \item Equal population standard deviations
                    \end{enumerate}
                    \item[Step 1] For a confidence level of $1-\alpha$, use a $t$-table to
                    find $t_{\alpha/2}$ with $df = n_1 + n_2 - 2$.
                    \item[Step 2] The endpoints of the confidence interval for $\mu_1-\mu_2$
                    are
                    \begin{center}
                        \[
                            (\bar{x_1}-\bar{x_2}) \pm t_{\alpha/2}*s_p\sqrt{(1/n_1)+(1/n_2)},    
                        \]
                    \end{center}   
                    where $s_p$ is the pooled sample standard deviation.
                    \item[Step 3] Interpret the confidence interval.
                    \item[Note:] The confidence interval is exact for normal populations and
                    is approximately correct for larger samples from nonnormal populations.  
                \end{description}

    \section*{Section 10.3}
    \noindent\rule{\textwidth}{0.4pt}
        \subsection*{Distribution of the Nonpooled $t$-Statistic}
            Suppose that $x$ is a normally distributed variable on each of two populations.
            Then, for independent samples of sizes $n_1$ and $n_2$ from the two populations,
            the variable
            \begin{center}
                \[
                    t = \frac{(\bar{x_1}-\bar{x_2}-(\mu_1-\mu_2))}
                             {\sqrt{(s_1^2/n_1)+(s_2^2/n_2)}}    
                \]
            \end{center}
            has approximately a $t$-distribution. The degrees of freedom used is obtained
            from the sample data. It is denoted $\Delta$ and given by
            \begin{center}
                \[
                    \Delta = \frac{[(s_1^2/n_1)+(s_2^2/n_2)]^2}
                                  {\frac{(s_1^2/n_1)^2}{n_1-1)}+\frac{(s_2^2/n_2)^2}{n_2-1}}   
                \]
            \end{center}
            rounded down to the nearest integer.
        \subsection*{Nonpooled $t$-Test}
            \begin{description}
                \item[Purpose] To perform a hypothesis test to compare population means,
                $\mu_1$ and $\mu_2$.
                \item[Assumptions] 
                \begin{enumerate}
                    \item Simple random sample
                    \item Independent samples 
                    \item Normal populations or large samples
                \end{enumerate}
                \item[Step 1] The null hypothesis is $H_0$: $\mu = \mu_0$, and the alternative
                hypothesis is
                \begin{center}
                    \[
                        H_a: \mu != \mu_0 \quad or \quad H_a: \mu < \mu_0 \quad or \quad 
                        H_a: \mu > \mu_0    
                    \]
                    \[
                        (Two \quad tailed)\quad\quad\quad(Left \quad Tailed)\quad\quad\quad
                        (Right \quad tailed)    
                    \]                    
                \end{center}
                \item[Step 2] Decide on the significance level, $\alpha$.
                \item[Step 3] Compute the value of the test statistic
                \begin{center}
                    \[
                        t = \frac{\bar{x_1}-\bar{x_2}}{\sqrt{(s_1^2/n_1)+(s_2^2/n_2)}},
                    \]
                \end{center} 
                Denote the value of the test statistic $t_0$. 
            \end{description}
            \subsubsection{Critical-Value Approach}
                \begin{description}
                    \item[Step 4] The critical value(s) are
                    \begin{center}
                        \[
                            \pm z _{\alpha/2} \quad\quad or \quad\quad -z_{\alpha} \quad\quad or
                            \quad\quad z_{\alpha}     
                        \]
                        \[
                            (Two \quad tailed)\quad\quad\quad(Left \quad tailed)\quad\quad\quad
                            (Right \quad tailed)
                        \]
                    \end{center}
                    with $df = \Delta$, where
                    \begin{center}
                        \[
                            \Delta = \frac{[(s_1^2/n_1)+(s_2^2/n_2)]^2}
                                          {\frac{(s_1^2/n_1)^2}{n_1-1)}+
                                          \frac{(s_2^2/n_2)^2}{n_2-1}}   
                        \]
                    \end{center}
                    rounded down to the nearest integer. Use a $t$-table to find the critical
                    value(s).
                    \item[Step 5] If the value of the test statistic falls in the rejection 
                    region, reject $H_0$; otherwise, do not reject $H_0$.
                    \item[Step 6] Interpret the results of the hypothesis test.            
                \end{description}
            \subsubsection{$P$-Value Approach}
                \begin{description}
                    \item[Step 4] The $t$-statistic has $df = \Delta$, where
                    \begin{center}
                        \[
                            \Delta = \frac{[(s_1^2/n_1)+(s_2^2/n_2)]^2}
                                          {\frac{(s_1^2/n_1)^2}{n_1-1)}+
                                          \frac{(s_2^2/n_2)^2}{n_2-1}}   
                        \]
                    \end{center}
                    rounded down to the nearest integer. Use a $t$-table to estimate the 
                    $P$-value, or obtain it exactly with using technology.
                    \item[Step 5] If $P \leq \alpha$, reject $H_0$; otherwise, do not reject
                    $H_0$.
                    \item[Step 6] Interpret the results of the hypothesis test. 
                \end{description}
        \subsection*{Nonpooled $t$-Interval Procedure}
            \begin{description}
                \item[Purpose] To find the confidence interval for the difference between two
                population means, $\mu_1$ and $\mu_2$.
                \item[Assumptions]
                \begin{enumerate}
                   \item Simple random sample
                   \item Independent samples
                   \item Normal populations or large samples 
                \end{enumerate} 
                \item[Step 1] For a confidence level of $1-\alpha$, use a $t$-table to find
                $t_{\alpha/2}$ with $df = \Delta$ , where
                \begin{center}
                    \[
                        \Delta = \frac{[(s_1^2/n_1)+(s_2^2/n_2)]^2}
                                      {\frac{(s_1^2/n_1)^2}{n_1-1)}+
                                      \frac{(s_2^2/n_2)^2}{n_2-1}}   
                    \]
                \end{center}
                rounded down to the nearest integer.
                \item[Step 2] The endpoints of the confidence interval for $\mu_1-\mu_2$ are
                \begin{center}
                    \[
                        (\bar{x_1}-\bar{x_2}) \pm t_{\alpha/2}*\sqrt{(s_1^2/n_1)+(s_2^2/n_2)}.    
                    \]
                \end{center} 
                \item[Step 3] Interpret the confidence interval.
            \end{description}
        \subsection*{Choosing between a Pooled and Nonpooled $t$-Procedure}    
            Suppose you want to use independent simple random samples to compare the means of
            two populations. To decide between a pooled $t$-procedure and a nonpooled
            $t$-procedure, follow these guidelines: If you are reasonably sure that the populations
            have nearly equal standard deviations, use a pooled $t$-procedure; otherwise, use a
            nonpooled $t$-procedure.
    
    \section*{Section 10.4}
    \noindent\rule{\textwidth}{0.4pt}
    

    \section*{Section 10.5}
    \noindent\rule{\textwidth}{0.4pt}


    \section*{Section 10.6}
    \noindent\rule{\textwidth}{0.4pt}


    \section*{Section 10.7}
    \noindent\rule{\textwidth}{0.4pt}

\end{document}