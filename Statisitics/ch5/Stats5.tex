\documentclass[12pt]{article}
\usepackage[]{fullpage}
\usepackage[]{amssymb}
\title{Chapter 5: Random Variables}
\author{}
\begin{document}
    \maketitle

    \noindent\rule{\textwidth}{0.4pt}
    \section*{Section 5.1}
    \noindent\rule{\textwidth}{0.4pt}
        \subsection*{Random Variable}
            A random variable is a quantitative variable whose value depends on chance.
        \subsection*{Discrete Random Variable}
            A discrete random variable is a random variable with only finite number of possible values, it
            usually involves a count of something.
        \subsection*{Probability Distribution and Probability Histogram}
            The probability distribution and probability histogram of a discrete variable show its 
            possible values and their likelihood.
            \begin{description}
                \item[Probability distribution:] A listing of the possible values and corresponding
                probabilities of discrete random variable, or a formula for the probabilities.
                \item[Probability histogram:] A graph of the probability distribution that displays the 
                possible values of a discrete random variable on the horizontal axis and the probability of
                those values on the vertical axis. The probability of each value is represented by a vertical
                bar whose height equals the probability.
            \end{description}
        \subsection*{Sum of the Probability of a Discrete Random Variable}
            For any discrete random variable X, we have $\sum P(X=x)=1$. The sum of the probabilities of
            the possible values of a discrete random variable equals 1.
        \subsection*{Interpretation of a Probability Distribution}
            In a large number of independent observations of a random variable X, the proportion of times
            each possible value occurs will approximate the probability distribution of X; or, equivalently,
            the proportion histogram will approximate the probability histogram for X.

    \noindent\rule{\textwidth}{0.4pt}
    \section*{Section 5.2}
    \noindent\rule{\textwidth}{0.4pt}
        \subsection*{Mean of a Discrete Random Variable (Expectation)}
            The mean of a discrete random variable X is denoted $\mu_x$ or, when no confusion will arise,
            simply $\mu$. It is defined by
            \begin{center}
                \[
                    \mu = \sum xP(X = x)    
                \]                  
            \end{center}
            The terms expected value and expectation are commonly used in place of the term mean.
        \subsection*{Interpretation of the Mean of a Random Variable (Expectation)}
            In a large number of independent observations of a random variable X, the average value of those
            observations will approximately equal the mean, $\mu$, of X. The larger the number of observations,
            the closer the average tends to be to $\mu$.
        \subsection*{Standard Deviation of a Discrete Random Variable}
            The standard deviation of a discrete random variable X is denoted $\sigma_X$ or, when no confusion
            will arise, simply $\sigma$. It is defined as
            \begin{center}
                \[
                    \sigma = \sqrt{\sum (x - \mu)^2 P(X = x)}    
                \]
            \end{center}
            The standard deviation of a discrete random variable can also be obtained from the computing 
            formula
            \begin{center}
                \[
                    \sigma = \sqrt{\sum x^2 P(X = x) - \mu^2}    
                \]
            \end{center}
            Roughly speaking, the standard deviation of a random variable $X$ indicates how far, on average,
            an observed value of X is from its mean. In particular, the smaller the standard deviation of X,
            the more likely it is that an observed value of $X$ will be close to its mean.
            \subsubsection*{Variance of X}
                The square of the standard deviation, $\sigma^2$ is called the variance of $X$. It is defined
                as 
                \begin{center}
                    \[
                        \sigma^2 = \sum (x - \mu)^2 P(X = x)    
                    \]
                \end{center}
                The computing formula for the variance of X is defined as
                \begin{center}
                    \[
                        \sigma^2 = \sum x^2 P(X = x) - \mu^2   
                    \]
                \end{center}

    \noindent\rule{\textwidth}{0.4pt}
    \section*{Section 5.3}
    \noindent\rule{\textwidth}{0.4pt}
        \subsection*{Binomial Coefficients}
            If $n$ is a positive integer and $x$ is a nonnegative integer less than or equal to n, then the
            binomial coefficient ${n \choose k}$ is defined as
            \begin{center}
                \[
                    {n \choose k} = \frac{n!}{k!(n-k)!}    
                \]             
            \end{center}
            The binomial coefficient is equivalent to $_nC_k$ and is often termed "n choose k".
        \subsection*{Bernoulli Trials}
            Bernoulli trials are identical and independent repetitions of an experiment with two possible
            outcomes. Repeated trials of an experiment are called Bernoulli trials if the following three 
            conditions are satisfied:
            \begin{enumerate}
                \item The experiment (each trial) has two possible outcomes, denoted generically $s$, for
                success, and $f$, for failure.
                \item The trials are independent, meaning that the outcome of one trial does not affect the
                outcome of other trials.
                \item The probability of a success, called the success probability and denoted $p$, remains
                the same from trial to trial.
            \end{enumerate}
        \subsection*{Binomial Distribution}
            The binomial distribution is the probability distribution for the number of successes in a
            sequence of Bernoulli trials.
        \subsection*{Number of Outcomes Containing a Specified Number of Successes}
            In n Bernoulli trials, the number of outcomes that contain exactly x successes equals the
            binomial coefficient ${n \choose k}$. In other words, there are ${n \choose k}$ ways of getting
            exactly x successes in n Bernoulli trials.
        \subsection*{Binomial Probability Formula}
            Let $X$ denote the total number of successes in $n$ Bernoulli trials with success probability
            $p$. The the probability distribution of the random variable $X$ is given by
            \begin{center}
                \[
                    P(X = x) = {n \choose x}p^x(1 - p)^{n-x}, x \in \mathbb{N}
                \]
            \end{center}
        \subsection*{Procedure to find a Binomial Probability Formula.}
            \begin{description}
                \item[Assumptions]
                \begin{enumerate}
                    \item $n$ trials are to be performed.
                    \item Two outcomes, success or failure, are possible for each trial.
                    \item The trials are independent.
                    \item The success probability, $p$, remains the same from trial to trial.                
                \end{enumerate}
                \item[Step 1] Identify a success.
                \item[Step 2] Determine $p$, the success probability.
                \item[Step 3] Determine $n$, the number of trials.
                \item[Step 4] The binomial probability formula for the number of successes, $X$, is
                \begin{center}
                    \[
                        P(X = x) = {n \choose x}p^x(1 - p)^{n-x}.    
                    \]
                \end{center}
            \end{description}
        \subsection*{Mean and Standard Deviation of a Binomial Random Variable}
            The mean and standard deviation of a binomial random variable with parameters $n$ and $p$ are
            \begin{center}
                \[
                    \mu = np and \sigma = \sqrt{np(1-p)},    
                \]                
            \end{center}
            respectively.
        \subsection*{Sampling and the Binomial Distribution}
            Suppose that a simple random sample of size $n$ is taken from a finite population in which the
            proportion of members that have a specified attribute
            \begin{itemize}
                \item has exactly a binomial distribution with parameters $n$ and $p$ if the sampling is
                done with replacement and
                \item has approximately a binomial distribution with parameters $n$ and $p$ if the
                sampling is done without replacement and the sample size does not exceed 5\% of the
                population size.
            \end{itemize}
            When a simple random sample is taken from a finite population, you can use a binomial
            distribution for the number of members obtained having a specified attribute,
            regardless of whether the sampling is with or without replacement, provided that, in the
            latter case, the sample size is small relative to population size.

    \noindent\rule{\textwidth}{0.4pt}
    \section*{Section 5.4}
    \noindent\rule{\textwidth}{0.4pt}
        \subsection*{Poisson Probability Formula}
            Probabilities for a random variable $X$ that has a Poisson distribution are given by the
            formula
            \begin{center}
                \[
                    P(X = x) = e^{-\lambda}\frac{\lambda^x}{x!},    x \in \mathbb{N}    
                \]                    
            \end{center}
            where $\lambda$ is a positive real number and $e$ is approximately 2.718. The random
            variable $X$ is called a Poisson random variable and is said to have the Poisson distribution
            with parameter $\lambda$.
            A Poisson random variable can be any integer value. Consequently, we cannot display all the
            probabilities for a Poisson random variable in a probability distribution table.
        \subsection*{Mean and Standard Deviation of a Poisson Random Variable}
            The mean and standard deviation of a Poisson random variable with parameter $\lambda$ are
            \begin{center}
                \[
                    \mu = \lambda and \sigma = \sqrt{\lambda}    
                \]                
            \end{center}
            respectively.
        \subsection*{To Approximate Binomial Probabilities by Using a Poisson Probability Formula}
            \begin{description}
                \item {Step 1} Find $n$, the number of trials, and $p$, the success probability.
                \item {Step 2} Continue only if $n \geq 100$ and $np \leq 10$.
                \item {Step 3} approximate the required binomial probabilities by using the Poisson
                probability formula
                \begin{center}
                    \[
                        P(X = x) = e^{-np}\frac{(np)^x}{x!}.   
                    \]                    
                \end{center}
            \end{description}

\end{document}




