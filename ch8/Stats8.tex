\documentclass[12pt]{article}
\usepackage[]{fullpage}
\title{Chapter 8: Notes}
\author{Evan Hunt}
\begin{document}
    \maketitle

    \noindent\rule{\textwidth}{0.4pt}
    \section*{Section 8.1}
    \noindent\rule{\textwidth}{0.4pt}
        \subsection*{Point Estimate}
            A point estimate of a parameter is the value of a statistic used to estimate the
            parameter. Roughly speaking, a point estimate of a parameter is our best guess for
            the value of the unknown parameter based on sample data. For instance, a sample mean
            is a point estimate of a population mean, and a sample standard deviation is a point
            estimate of a population standard deviation.
        \subsection*{Confidence-Interval Estimate}
            A confidence-interval estimate for a parameter provides a range of numbers along with
            a percentage confidence that the parameter lies in that range.
        \begin{description}
            \item[Confidence interval (CI):] An interval of numbers obtained from a point
            estimate of a parameter.
            \item[Confidence level:] The confidence we have that the parameter lies in the
            confidence interval (i.e., that the confidence interval contains the parameter). 
            \item[Confidence-interval estimate:] The confidence level and confidence 
            interval. 
        \end{description}
        
    \section*{Section 8.2}
    \noindent\rule{\textwidth}{0.4pt}
        \subsection*{One-Mean z-Interval Procedure}
            \begin{description}
                \item[Purpose] To find a confidence interval for a population mean, $\mu$.
                \item[Assumptions]
            \end{description}
            \begin{enumerate}
                \item Simple random sample
                \item Normal population or large sample size
                \item $\sigma$ is known                    
            \end{enumerate}
            \begin{description}
                \item[Step 1] For a confidence level of $1-\alpha$, use a $z$-table to find
                $z_{\alpha/2}$.
                \item[Step 2] The confidence interval for $\mu$ is from
                \begin{center}
                    \[
                        \bar{x}-z_{\alpha/2}*\frac{\sigma}{\sqrt{n}} to 
                        \bar{x}+z_{\alpha/2}*\frac{\sigma}{\sqrt{n}}
                    \]
                \end{center}
                where $z_{\alpha/2}$ is found in Step 1, $n$ is the sample size, and $\bar{x}$
                is computed from the sample data.
                \item[Step 3] Interpret the confidence interval.                    
            \end{description}  
            \begin{description}
                \item[Note:] The confidence interval is exact for normal populations and is
                approximately correct for large samples from nonnormal populations. 
            \end{description}

        \subsection*{When to Use the One-Mean $z$-Interval Procedure}
            \begin{itemize}
                \item For small samples, less than 15 roughly, the $z$-interval procedure should
                be used only when the variable under consideration is normally distributed or 
                very close to being so.
                \item For samples of moderate size, between 15 and 30 roughly, the $z$-interval
                procedure can be used unless the data contain outliers or the variable under
                consideration is far from being normally distributed.
                \item For large samples, around 30 or more, the $z$-interval procedure can be
                used without restriction, generally. However, if outliers are present and their
                removal is not justified, you should compare the confidence intervals obtained 
                with and without the outliers to see what effect the outliers have. If the effect
                is substantial, use a different procedure or take another sample, if possible.                   
                \item If outliers are present but their removal is justified and results in a
                data set for which the $z$-interval procedure is approximate (as previously 
                stated), the procedure can be used.
            \end{itemize}
            \subsubsection*{A Fundamental Principle of Data Analysis}
                Before performing a statistical-inference procedure, examine the sample data. If
                any of the conditions required for using the procedure appear to be violated, do
                not apply the procedure. Instead use a different, more appropriate procedure, if
                one exists.
        \subsection*{Margin of Error for the Estimate of $\mu$}
            The margin of error for the estimate of $\mu$ is $z_{\alpha/2}*\sigma/\sqrt{n}$,
            which is denoted by the letter $E$. Thus,
            \begin{center}
                \[
                    E = z_{\alpha/2}*\sigma/\sqrt{n}.    
                \]                
            \end{center}
            The margin of error for the estimate of population mean indicates the accuracy with
            which a sample mean estimates the unknown population mean. 
        \subsection*{Confidence and Accuracy}
            For a fixed sample size, decreasing the confidence level decreases the margin of
            error and, hence, improves the accuracy of a confidence-interval estimate.
        \subsection*{Sample Size and Accuracy}
            For a fixed confidence level, increasing the sample size decreases the margin of
            error and, hence, improves the accuracy of a confidence-interval estimate.
        \subsection*{Sample Size for Estimating $\mu$}
            The sample size required for a $(1-\alpha)$-level confidence interval for $\mu$ with
            a specified margin of error, $E$, is given by the formula
            \begin{center}
                \[
                n = (\frac{z_{\alpha/2}*\sigma}{E})^2,
                \]
            \end{center}
            rounded up to the nearest whole number.

    \section*{Section 8.3}
    \noindent\rule{\textwidth}{0.4pt}
        \subsection*{Studentized Version of the Sample Mean}
            Suppose that a variable $x$ of a population is normally distributed with mean $\mu$.
            Then, for samples of size $n$, the variable
            \begin{center}
                \[
                    t = \frac{\bar{x}-\mu}{s/\sqrt{n}}    
                \]  
            \end{center}
            has the $t$-distribution with $n-1$ degrees of freedom (df). For a normally
            distributed variable, the studentized version of the sample mean has the
            $t$-distribution with degrees of freedom 1 less that the sample size.
        \subsection*{Basic Properties of $t$-Curves}
            \begin{description}
                \item[Property 1:] The total area under a t-curve equals 1.
                \item[Property 2:] A t-curve extends indefinitely in both directions,
                approaching, but never touching, the horizontal axis as it does so.
                \item[Property 3:] A t-curve is symmetric about 0.
                \item[Property 4:] As the number of degrees of freedom becomes larger,
                $t$-curves look increasingly like the standard normal curve.   
            \end{description}
        \subsection*{One-Mean $t$-Interval Procedure}
            \begin{description}
                \item[Purpose] To find a confidence interval for population mean, $\mu$
                \item[Assumptions]
                \begin{enumerate}
                    \item Simple random sample
                    \item Normal population or large sample
                    \item $\sigma$ unknown                    
                \end{enumerate}  
                \begin{description}
                    \item[Step 1] For a confidence level of $(1-\alpha)$, use a $t$-table to find 
                    $t_{\alpha/2}$ with $df = n-1$, where $n$ is the sample size.
                    \item[Step 2] The confidence interval for $\mu$ is from
                    \begin{center}
                        \[
                            \bar{x} - t_{\alpha/2}*\frac{s}{\sqrt{n}} to
                            \bar{x} + t_{\alpha/2}*\frac{s}{\sqrt{n}},    
                        \]                        
                    \end{center} 
                    where $t_{\alpha/2}$ is found in Step 1 and $\bar{x}$ and s are computed from
                    sample data.
                    \item[Step 3] Interpret the confidence interval.
                    \item[Note:] The confidence interval is exact for normal populations and is 
                    approximately correct for large samples from nonnormal populations.
                \end{description}
            \end{description}

\end{document}



